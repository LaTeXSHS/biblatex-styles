\documentclass{beamer}
\usepackage{fontspec,polyglossia,xunicode,hyperref}
\setmainfont{Linux Libertine O}
\setmainlanguage{french}

\usepackage{minted}
\newcounter{code}
\renewcommand{\thecode}{\arabic{code}.~}
\newcommand{\code}[3]{\only<#1>{\stepcounter{code}\begin{block}{\thecode#2}\footnotesize\inputminted{latex}{code/#3.tex}\end{block}}}

\usepackage{tikz}

\author{Maïeul Rouquette}
\date{Rencontres \LaTeX\ et SHS}
\title{Personnaliser les styles Bib\LaTeX}
\institute{Université de Lausanne --- IRSB}

\usetheme{Darmstadt}
\begin{document}


\begin{frame}
	\titlepage
	\vfill
	{\tiny Licence Creative Commons France 3.0 - Paternité - Partage à l'identique}
\end{frame}

\begin{frame}
	\frametitle{Deux manières différents}
	\begin{description}
		\item[\alert<1>{Manière rapide}] Dans un fichier .tex chargé dans le préambule
		\item[\alert<2>{Manière propre}] Dans un jeu de styles redistribuables : .cbx, .bbx, .lbx
	\end{description}
\end{frame}

\begin{frame}
	\frametitle{S'inspirer des styles par défaut}
	\begin{enumerate}
		\item<1->Dossier texmf-dist/tex/latex/biblatex (dans \TeX Live)
		\item<2->Type de fichiers :
			\begin{description}
				\item<3->[.def] Réglages communs à l'ensemble des styles
				\item<4->[.bbx] Bibliographie finale
				\item<5->[.cbx] Styles pour les citations dans le corps du texte (ou en note)
				\item<6->[.lbx] Chaînes de langues
			\end{description}
	\end{enumerate}
\end{frame}

\begin{frame}
	\frametitle{Plusieurs niveaux d'adaptation}
	\begin{overprint}
		\only<1>{\footnotesize\begin{center}
\begin{tikzpicture}[edge from parent path={[->,thick]
      (\tikzparentnode) -- (\tikzchildnode)},grow=right,level 1/.style={sibling distance=9em,level distance=.5cm},level 2/.style={sibling distance=9em,level distance=3cm},level 3/.style={sibling distance=9em,level distance=4cm},
	every node/.style={align=center}]
	\node {Commande de citation} 
		child { node {Macros bibliographiques}}
		child { node {Driver bibliographique}
			child { node {Macros bibliographiques}
			 	child {
					node{Impression de champs}
					}
			 	child {
					node{Appel à d'autre macros}
					}
				child {
					node{Appel à des \\ chaînes de langues}
					}
				}
			child { node {Séparateur\\ d'unité bibliographique}}
			}	
	;		
\end{tikzpicture}
\end{center}}
		\code{2}{Commandes Bib\LaTeX\ à redéfinir}{commandes}
		\code{3}{Chaînes de langues à redéfinir}{chaines}
		\code{4}{Formatage des champs}{champs}
		\code{5}{Formatage des noms}{noms}
		\code{6}{Formatage des listes}{listes}
		\code{7}{Macros}{macros}
		\code{8}{Drivers}{drivers}
	\end{overprint}
\end{frame}
\end{document}