\documentclass{beamer}
\usepackage{fontspec,polyglossia,xunicode,hyperref,csquotes}
\setmainfont{Linux Libertine O}
\setmainlanguage{french}
\setotherlanguage{english}
\usepackage{minted}
\newcounter{code}
\renewcommand{\thecode}{\arabic{code}.~}
\newenvironment{codes}{\only<1>{\setcounter{code}{0}}\begin{overprint}}{\end{overprint}}
\newcommand{\code}[2]{%
	\only<+>{%
		\stepcounter{code}%
		\begin{block}{\thecode#1}%
			\begin{english}\footnotesize%
				\inputminted{latex}{code/#2.tex}%
			\end{english}%
			\end{block}%
			}%
	}
\newcommand{\alertdesc}[2]{\item[\alert<+>{#1}]#2}
\usepackage{tikz}

\usepackage{biblatex-philologue/biblatex-philologue}
\bibliography{biblatex-styles.bib}
\renewbibmacro*{cite}{%
  \usebibmacro{cite:citepages}%
    {\usebibmacro{cite:full}}}
    
\author{Maïeul Rouquette}
\date{Rencontres \LaTeX\ et SHS}
\title{Personnaliser les styles Bib\LaTeX}
\institute{Université de Lausanne --- IRSB}

\usetheme{Darmstadt}
\begin{document}


\begin{frame}
	\titlepage
	\vfill
	{\tiny Licence Creative Commons France 3.0 - Paternité - Partage à l'identique}
\end{frame}

\begin{frame}
	\frametitle{Deux manières différents}
	\begin{description}
		\alertdesc{Manière rapide}{Dans un fichier .tex chargé dans le préambule}
		\alertdesc{Manière propre}{Dans un jeu de styles redistribuables : .cbx, .bbx, .lbx}
	\end{description}
\end{frame}

\begin{frame}
	\frametitle{S'inspirer des styles par défaut}
	\begin{enumerate}
		\item<1->Dossier texmf-dist/tex/latex/biblatex (dans \TeX Live)
		\item<2->Type de fichiers :
			\begin{description}
				\item<+->[.def] Réglages communs à l'ensemble des styles
				\item<+->[.bbx] Bibliographie finale
				\item<+->[.cbx] Styles pour les citations dans le corps du texte (ou en note)
				\item<+->[.lbx] Chaînes de langues
			\end{description}
	\end{enumerate}
\end{frame}

\begin{frame}
	\frametitle{Notions de bases}
	\only<1>{\footnotesize\begin{center}
\begin{tikzpicture}[edge from parent path={[->,thick]
      (\tikzparentnode) -- (\tikzchildnode)},grow=right,level 1/.style={sibling distance=9em,level distance=.5cm},level 2/.style={sibling distance=9em,level distance=3cm},level 3/.style={sibling distance=9em,level distance=4cm},
	every node/.style={align=center}]
	\node {Commande de citation} 
		child { node {Macros bibliographiques}}
		child { node {Driver bibliographique}
			child { node {Macros bibliographiques}
			 	child {
					node{Impression de champs}
					}
			 	child {
					node{Appel à d'autre macros}
					}
				child {
					node{Appel à des \\ chaînes de langues}
					}
				}
			child { node {Séparateur\\ d'unité bibliographique}}
			}	
	;		
\end{tikzpicture}
\end{center}}
	\only<2-5>{
		\begin{description}
			\alertdesc{Driver}{Description du formatage d'un type d'entrée}
			\alertdesc{Macro}{Sous description de formatage. Bien souvent correspond à une ou plusieurs unité.}
			\alertdesc{Unité}{Sous division d'une référence bibliographique. Bien souvent séparé par des signes bibliographiques}
			\alertdesc{Cmd. de ponctuation}{Commande spécifique à Bib\LaTeX\ permettant d'afficher un signe de ponctuation tout en prévenant le doublement}
		\end{description}
	}
	\only<6->{
		\begin{description}
			\alertdesc{Name}{Tout champ correspondant à une personne morale ou physique}
			\alertdesc{List}{Tout champ pouvant contenir des valeurs séparés par \enquote{and} (sauf nom)}
			\alertdesc{Field}{Tout champ autre que liste et noms}
		\end{description}
	}
\end{frame}
\begin{frame}
	\frametitle{Alias}
	\begin{codes}
		\code{Entre fichiers}{alias-fichiers}
		\code{Entre formatage de champs, noms, listes}{alias-champ-formatage}
		\code{Entre formatage de type}{alias-type-formatage}
		\code{Entre affection de de type champs (Biber)}{alias-affectation}
	\end{codes}
\end{frame}
\begin{frame}
	\frametitle{Plusieurs niveaux d'adaptation}
	\begin{codes}
		\code{Commandes BibLaTeX à redéfinir}{chaines}
		\code{Chaînes de langues à redéfinir}{chaines}
		\code{Formatage des champs}{champs}
		\code{Formatage des noms}{noms}
		\code{Formatage des listes}{listes}
		\code{Macros}{macros}
		\code{Drivers}{drivers}
	\end{codes}
\end{frame}
\begin{frame}
	\frametitle{Quelques commandes utiles}
	\begin{codes}
		\code{Imprimer field / liste / name}{imprimer-champs}
		\code{Tester la présence d'un champ}{iffieldundef}
		\code{Tester la valeur d'un champ}{iffieldequals}
		\code{Tester le type d'entrée}{ifentrytype}
		\code{Tester si l'entrée a déjà été citée}{ifciteseen}
		\code{Tester si l'entrée est la seule d'un auteur}{ifsingletitle}
		\only<+>{La liste est longue \ldots voir la documentation}
	\end{codes}
\end{frame}
\begin{frame}
	\frametitle{Pour aller plus loins\ldots}
	\begin{itemize}
		\item<+->\cite{biblatex_doc}
		\item<+->Un peu de narcissisme :
		\begin{itemize}
			\item\cite{Rouquette2012}
			\item\cite{Rouquette_geek_biblatex}
		\end{itemize}
	\end{itemize}
\end{frame}
\end{document}