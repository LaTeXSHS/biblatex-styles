\documentclass{beamer}
\usepackage{fontspec,polyglossia,xunicode,hyperref}
\setmainfont{Linux Libertine O}
\setmainlanguage{french}
\usepackage{minted}
\newcommand{\code}[3]{\only<#1>{\begin{block}{#2}\inputminted{latex}{code/#3.tex}\end{block}}}
\author{Maïeul Rouquette}
\date{Rencontres \LaTeX\ et SHS}
\title{Personnaliser les styles biblatex}
\institute{Université de Lausanne --- IRSB}

\usetheme{Darmstadt}
\begin{document}


\begin{frame}
	\titlepage
	\vfill
	{\tiny Licence Creative Commons France 3.0 - Paternité - Partage à l'identique}
\end{frame}

\begin{frame}
	\frametitle{Deux manières différents}
	\begin{description}
		\item[\alert<1>{Manière rapide}] Dans un fichier .tex chargé dans le préambule
		\item[\alert<2>{Manière propre}] Dans un jeu de styles redistribuables : .cbx, .bbx, .lbx
	\end{description}
\end{frame}

\begin{frame}
	\frametitle{S'inspirer des styles par défaut}
	\begin{enumerate}
		\item<1->Dossier texmf-dist/tex/latex/biblatex (dans \TeX Live)
		\item<2->Type de fichiers :
			\begin{description}
				\item<3->[.def] Réglages communs à l'ensemble des styles
				\item<4->[.bbx] Bibliographie finale
				\item<5->[.cbx] Styles pour les citations dans le corps du texte (ou en note)
				\item<6->[.lbx] Chaînes de langues
			\end{description}
	\end{enumerate}
\end{frame}
\end{document}